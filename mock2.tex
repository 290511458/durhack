\documentclass[10pt]{article}

\usepackage{enumerate,graphicx,tikz,pgfpages}
\usepackage[final]{pdfpages}
\usepackage{amssymb}

\pagestyle{empty}

\begin{document}

\begin{center}
{\large \bf Theory of Computation\\[3pt]  Mock benchtest}
\end{center}

\bigskip 

\begin{center}
{\large \bf Part I: Models of Computation}
\end{center}

% ***********************************************************

\noindent {\bf \underline{Question 1}}

\bigskip \noindent  Which of the following two assertions about a Turing Machine (TM) is true? 
\begin{enumerate} 
\item[(i)] The input alphabet can be the same as the tape alphabet. 
\item[(ii)] At a given moment, there could be infinitely many symbols 
from the input alphabet onto the tape. 
\end{enumerate}

\begin{enumerate}[{\bf (A)}]
\addtolength{\itemsep}{-5pt}
\item The first only. 
\item The second only. 
\item Both of them. 
\item Neither of them. 
\end{enumerate}
%Answer D from the definition

\bigskip

% ***********************************************************

\noindent {\bf \underline{Question 2}}

\bigskip \noindent  Consider the complement of the Halting Problem ($H$), $\mathrm{co-}H$: 
given a TM ${\mathcal M}$ and an input $w$, is it the case that ${\mathcal M}$ 
doesn't terminate on input $w$? Which of the following is true? 

\begin{enumerate}[{\bf (A)}]
\addtolength{\itemsep}{-5pt} 
\item This problem is not semi-decidable. 
\item This problem is semi-decidable but not decidable. 
\item This problem is decidable. 
\item None of them. 
\end{enumerate}
%Answer A: if it were semi-decidable and given than H is semi-decidable, H would be decidable.

\bigskip

% ***********************************************************

\noindent {\bf \underline{Question 3}}

\bigskip \noindent  What can you say about reducibilities between $H$ and $\mathrm{co-}H$?

\begin{enumerate}[{\bf (A)}]
\addtolength{\itemsep}{-5pt} 
\item $H$ is m-reducible to $\mathrm{co-}H$. 
\item $\mathrm{co-}H$ is m-reducible to $H$.
\item Neither is true.
\item Both are true. 
\end{enumerate}

The same question but this time with T-reduction, instead of m-reduction.
% Answer C for m-reduction and D for T-reduction.

\bigskip

% ***********************************************************

\noindent {\bf \underline{Question 4}}

\bigskip \noindent  What is the sequence encoded by the G{\"o}del Number $600$?

\begin{enumerate}[{\bf (A)}]
\addtolength{\itemsep}{-5pt} 
\item $3, 1, 1$. 
\item $2, 0, 1$.
\item $2, 3, 5$. 
\item None of them. 
\end{enumerate}
% Answer A. We defined the G{\"o}del Number of a sequence $x_1,\dots x_{n-1},x_{n}$ to be
% $p_1^{x_1}\dots p_{n-1}^{x_{n-1}} p_n^{x_{n}+1}$. 

\newpage

\noindent {\bf \underline{Question 5}}

\bigskip \noindent  Which of the following has a proof?
\begin{enumerate}[{\bf (A)}]
\addtolength{\itemsep}{-5pt} 
\item Every positive instance of the Halting Problem $H$. 
\item Some negative instances of $H$.
\item All negative instances of $\mathrm{co-}H$.
\item All of the above. 
\item None of the above. 
\end{enumerate}
% Answer D. (A) and (C) are the same and a simulation of the run is a proof. Trivial TMs that go straight into an infinite loop ignoring the input have proof for non-termination.

\bigskip
\begin{center}
{\large \bf Part II: Algorithms and Complexity}
\end{center}

\bigskip
\noindent 
{\bf \underline{Question 6}}

\bigskip 
\noindent
Let $f(n)$ and $g(n)$ be two positive functions defined on the
natural numbers. Suppose there exist constants $c>0$ and $n_0$ such
that $f(n_0) \leq cg(n_0)$. Which statement below can we deduce from
this?

\begin{enumerate}[{\bf (A)}]
\addtolength{\itemsep}{-5pt}
\item $f(n)=O(g(n))$
\item $f(n)=\Omega(g(n))$
\item $f(n)=\Theta(g(n))$
\item Each of the above can be deduced.
\item None of the above can be deduced.
\end{enumerate}
%Answer: E

\bigskip
\noindent 
{\bf \underline{Question 7}}

\bigskip 
\noindent
Let $T(n)=8T(\frac{n}{2})+n^2$ (where we interpret $\frac{n}{2}$ to
mean $\lfloor \frac{n}{2}\rfloor$). Which statement below is true?

\begin{enumerate}[{\bf (A)}]
\addtolength{\itemsep}{-5pt}
\item $T(n)=\Theta(n^2)$.
\item $T(n)=\Theta(n^3)$.
\item $T(n)=\Theta(n^3\mbox{lg}\,n)$.
\item $T(n)=\Theta(n^4)$.
\item None of the above is true.
\end{enumerate}
%Answer: B

\newpage
\noindent
{\bf \underline{Question 8}}

\bigskip
\noindent
Let $A$ be a $3\times 3$ matrix. Let $B$ be a $3\times 1$ matrix. Let $C$ be a $3\times 3$ matrix.
Which statement below is true?
\begin{enumerate}[{\bf (A)}]
\addtolength{\itemsep}{-5pt}
\item The number of ways to parenthesize $ABC$ is $27$.
\item The number of ways to parenthesize $ABC$ is $28$.
\item The number of ways to parenthesize $ABC$ is $1$.
\item It is not possible tot compute the matrix product $ABC$. 
\item None of the above.
\end{enumerate}
%Answer: D

\bigskip
\noindent 
{\bf \underline{Question 9}}

\bigskip 
\noindent
Which problem below can be solved by a greedy algorithm?
\begin{enumerate}[{\bf (A)}]
\addtolength{\itemsep}{-5pt}
\item Longest Common Subsequence.
\item 0-1 knapsack.
\item Fractional knapsack.
\item Rod cutting.
\item None of the above.
\end{enumerate}
%Answer: C

\bigskip
\noindent 
{\bf \underline{Question 10}}

\bigskip \noindent
For $r\geq 1$, let $G$ be the $(r+1)$-vertex star  with vertices $u$, $v_1,\ldots,v_r$ and edges $uv_1,\ldots,uv_r$. Which statement below is true?
\begin{enumerate}[{\bf (A)}]
\addtolength{\itemsep}{-5pt}
\item $G$ is $P_4$-free, but not a cograph.
\item $G$ is a cograph but, depending on $r$, may not always be a $P_4$-free graph.
\item $G$ is a cograph and a $P_4$-free graph.
\item None of the above.
\end{enumerate}
%Answer: C

\end{document}
